% !TeX TXS-program:compile = txs:///arara
% arara: pdflatex: {shell: yes, synctex: no, interaction: batchmode}
% arara: pdflatex: {shell: yes, synctex: no, interaction: batchmode} if found('log', '(undefined references|Please rerun|Rerun to get)')

\documentclass[a4paper]{article}
\usepackage[svgnames]{xcolor}
\usepackage[french]{babel}
\usepackage[utf8]{inputenc}
\usepackage[T1]{fontenc}
\usepackage{WriteOnGrid}
\usetikzlibrary{decorations.pathreplacing}
\usepackage{amsmath,amssymb}
\usepackage{fontawesome5}
\usepackage{enumitem}
\usepackage{frcursive}
\usepackage{tabularray}
\usepackage{fancyvrb}
\usepackage{fancyhdr}
\fancyhf{}
\renewcommand{\headrulewidth}{0pt}
\lfoot{\sffamily\small [WriteOnGrid]}
\cfoot{\sffamily\small - \thepage{} -}
\rfoot{\hyperlink{matoc}{\small\faArrowAltCircleUp[regular]}}

%\usepackage{hvlogos}
\usepackage{hologo}
\providecommand\tikzlogo{Ti\textit{k}Z}
\providecommand\TeXLive{\TeX{}Live\xspace}
\providecommand\PSTricks{\textsf{PSTricks}\xspace}
\let\pstricks\PSTricks
\let\TikZ\tikzlogo
\newcommand\TableauDocumentation{%
	\begin{tblr}{width=\linewidth,colspec={X[c]X[c]X[c]X[c]X[c]X[c]},cells={font=\sffamily}}
		{\huge \LaTeX} & & & & &\\
		& {\huge \hologo{pdfLaTeX}} & & & & \\
		& & {\huge \hologo{LuaLaTeX}} & & & \\
		& & & {\huge \TikZ} & & \\
		& & & & {\huge \TeXLive} & \\
		& & & & & {\huge \hologo{MiKTeX}} \\
	\end{tblr}
}

\usepackage{hyperref}
\urlstyle{same}
\hypersetup{pdfborder=0 0 0}
\usepackage[margin=1.5cm]{geometry}
\setlength{\parindent}{0pt}
\definecolor{LightGray}{gray}{0.9}

\def\TPversion{0.1.0}
\def\TPdate{16 Janvier 2023}

\usepackage[most]{tcolorbox}
\tcbuselibrary{minted}
\NewTCBListing{PresentationCode}{ O{blue} m }{%
	sharp corners=downhill,enhanced,arc=12pt,skin=bicolor,%
	colback=#1!1!white,colframe=#1!75!black,colbacklower=white,%
	attach boxed title to top right={yshift=-\tcboxedtitleheight},title=Code \LaTeX,%
	boxed title style={%
		colframe=#1!75!black,colback=#1!15!white,%
		,sharp corners=downhill,arc=12pt,%
	},%
	fonttitle=\color{#1!90!black}\itshape\ttfamily\footnotesize,%
	listing engine=minted,minted style=colorful,
	minted language=tex,minted options={tabsize=4,fontsize=\footnotesize,autogobble},
	#2
}

\newcommand\Cle[1]{{\bfseries\sffamily\textlangle #1\textrangle}}

\begin{document}

\pagestyle{fancy}

\thispagestyle{empty}

\vspace{2cm}

\begin{center}
	\begin{minipage}{0.75\linewidth}
	\begin{tcolorbox}[colframe=yellow,colback=yellow!15]
		\begin{center}
			\begin{tabular}{c}
				{\Huge \texttt{WriteOnGrid [fr]}}\\
				\\
				{\LARGE Pour écrire sur les} \\
				{\LARGE lignes d'une grille.}
			\end{tabular}
			
			\medskip
			
			{\small \texttt{Version \TPversion{} -- \TPdate}}
		\end{center}
	\end{tcolorbox}
\end{minipage}
\end{center}

\vspace{0.5cm}

\begin{center}
	\begin{tabular}{c}
	\texttt{Cédric Pierquet}\\
	{\ttfamily c pierquet -- at -- outlook . fr}\\
	\texttt{\url{https://github.com/cpierquet/WriteOnGrid}}
\end{tabular}
\end{center}

\vspace{0.5cm}

{$\blacktriangleright$~~Quelques commandes créer une grille (5x5 ou Seyes) et écrire \og sur \fg{} les lignes.}

\smallskip

{$\blacktriangleright$~~Personnalisation de la taille de la grille, des marges, etc.}

\vspace{1cm}

\begin{center}
	\begin{EnvQuadrillage}[NbCarreaux=22x8]
	\EcrireLigne{mon texte sur la ligne 1\ldots}
	\EcrireLigne<center>{mon texte centré sur la ligne 2\ldots}
	\EcrireLigne<right>{mon texte aligné à droite sur la ligne 3\ldots}
	\EcrireLigne[DecalH=2]{mon texte décalé de 2 carreaux sur la ligne 4\ldots}
	\PasseLigne
	\EcrireLigne{\sffamily mon texte sur la ligne 6\ldots}
	\EcrireLigne[DecalH=-1]{\small\cursive mon texte, décalé à gauche de 1 carreau, sur la ligne 7\ldots}
\end{EnvQuadrillage}
\end{center}

\begin{EnvQuadrillage}[NbCarreaux=24x5,Marge=1,Elargir=2/2,Seyes]
	\EcrireLigne[Echelle=1.5]{\textcolor{red}{mon texte sur la ligne 1\ldots}}
	\EcrireLigne[Echelle=1.5]{\textcolor{blue}{mon texte sur la ligne 2\ldots}}
	\EcrireLigne[Echelle=1.5,DecalH=-1]{$1+\frac{1}{2}=\frac32$ et $(1+x)^2=1+2x+x^2$ sur la ligne 3\ldots}
	\EcrireLigne[Echelle=1.5]{\cursive mon texte sur la ligne 4\ldots}
\end{EnvQuadrillage}

\vspace{0.5cm}

%\hfill{}\textit{Merci à Denis Bitouzé et à Patrick Bideault pour leurs retours et idées !}

\vfill

\hrule

\medskip

\TableauDocumentation

\medskip

\hrule

\medskip

\newpage

\part*{Introduction}

\section{Le package writeongrid}

\subsection{Chargement du package, packages utilisés}

Le package \textsf{writeongrid} se charge dans le préambule via la commande :

\begin{PresentationCode}{listing only}
\usepackage{writeongrid}
\end{PresentationCode}

Il est compatible avec les compilations usuelles en \textsf{latex}, \textsf{pdflatex}, \textsf{lualatex} ou \textsf{xelatex}.

\medskip

Il charge les packages et librairies suivantes :

\begin{itemize}
	\item \texttt{tikz} avec les librairies \Cle{calc} et \Cle{positionning} ;
	\item \texttt{xstring}, \texttt{xparse} et  \texttt{simplekv}.
\end{itemize}

\subsection{\og Philosophie \fg{} du package}

L'idée est de proposer, grâce à \TikZ, des \textsf{commandes} et \textsf{environnements} pour travailler sur un quadrillage et écrire \og sur les lignes \fg.

\begin{PresentationCode}{listing only}
%environnement francisé, avec clés en français pour préparer la grille
%commandes pour placer ou passer une ligne

\begin{EnvQuadrillage}[clés]<couleur>
	\EcrireLigne[clés]<alignement>{texte}
	\PasseLigne
\end{EnvQuadrillage}
\end{PresentationCode}

\subsection{Fonctionnement global}

La grille est créée en spécifiant le nombre de carreaux (sous la forme nbCol$\times$nbLig), et on peut ensuite spécifier des \textit{débordements} latéraux pour éventuellement étendre le quadrillage dans les marges (gauche et droite). On peut également \textit{jouer} sur la marge, pour commencer les lignes à un niveau donné.

\medskip

Ci-dessous on représente une grille $5\times5$ :

\begin{itemize}
	\item de taille 24x5 ;
	\item avec un élargissement de \textcolor{red}{2 carreaux à gauche} et \textcolor{blue}{3 carreaux à droite} ;
	\item en commençant à écrire au niveau du \textcolor{orange}{1\up{er} carreau} ;
	\item un \textcolor{ForestGreen}{\textit{cadre}} est rajouté pour visualiser la grille \textit{de base}.
\end{itemize}

\medskip

\begin{tikzpicture}
	\useasboundingbox (0,0) rectangle ({0.5*24},{-0.5*5}) ;
	\draw[xstep=0.5,ystep=0.5,thin,lightgray!75] ({-0.5*2},0) grid ({0.5*24+0.5*3},{-0.5*5}) ;
	\draw[thick,decorate,decoration={brace,amplitude=8pt,mirror}](0,{-0.5*5-0.25})--({0.5*24},{-0.5*5-0.25}) node[midway,below=8pt,font=\small\sffamily] {24C} ;
	\draw[red,thick,decorate,decoration={brace,amplitude=8pt,mirror}] ({-2*0.5},{-0.5*5-0.25})--({0},{-0.5*5-0.25}) node[midway,below=8pt,font=\small\sffamily] {2C} ;
	\draw[blue,thick,decorate,decoration={brace,amplitude=8pt,mirror}] ({0.5*24},{-0.5*5-0.25})--({0.5*24+3*0.5},{-0.5*5-0.25}) node[midway,below=8pt,font=\small\sffamily] {3C} ;
	\draw[thick,decorate,decoration={brace,amplitude=8pt}] ({0.5*24+3*0.5+0.25},0)--({0.5*24+3*0.5+0.25},{-0.5*5}) node[midway,right=8pt,font=\small\sffamily] {5C} ;
	\draw[very thick,orange,densely dashed] ({0.5*1},0)--({0.5*1},{-0.5*5}) ;
	\draw[ForestGreen,thick] (0,0) rectangle ({0.5*24},{-0.5*5}) ;
\end{tikzpicture}

\vspace{1.5cm}

Il est à noter que le figure \texttt{tikzpicture} est \textit{délimitée} par le \textcolor{ForestGreen}{\textit{cadre}}, afin de pouvoir gérer les débordements et l'alignement de l'environnement !

\medskip

De plus, le bord gauche du \textcolor{ForestGreen}{\textit{cadre}} est aligné sur la marge gauche de la page, donc la position du quadrillage dépend en partie de la configuration de \texttt{\textbackslash parindent}.

\pagebreak

\section{Commandes, clés et options}

\subsection{L'environnement}

\begin{PresentationCode}{listing only}
%environnement francisé, avec clés en français pour préparer la grille

\begin{EnvQuadrillage}[clés]<couleur>
	...
\end{EnvQuadrillage}
\end{PresentationCode}

Le premier argument, \textit{optionnel}, entre \texttt{[...]} propose les \Cle{clés} :

\begin{itemize}
	\item \Cle{NbCarreaux} pour spécifier le nombre de carreaux, sous la forme \texttt{(nbCol)x(nbLig)} ; \hfill~défaut : \Cle{17x5}
	\item \Cle{Unite} pour spécifier l'échelle de la figure ; \hfill~défaut : \Cle{1}
	\item \Cle{Marge} pour spécifier la \textcolor{orange}{marge} du début des lignes ; \hfill~défaut : \Cle{0}
	\item \Cle{Elargir} pour préciser les carreaux de débordements, sous la forme unique \texttt{\textcolor{red}{G}\textcolor{blue}{D}} ou par côté \texttt{\textcolor{red}{G}/\textcolor{blue}{D}} ;\hfill~défaut : \Cle{0}
	\item le booléen \Cle{Cadre} pour afficher le cadre de base du quadrillage ;\hfill~défaut : \Cle{false}
	\item le booléen \Cle{Seyes} pour afficher un quadrillage de type Seyes.\hfill~défaut : \Cle{false}
\end{itemize}

Le second argument, \textit{optionnel}, entre \texttt{<...>} est quant à lui la couleur de base du quadrillage, \Cle{lightgray} par défaut.

\medskip

\begin{PresentationCode}{listing only}
%les cadres ont été rajoutés pour la sortie

%30x5 petits carreaux, sans dépassement, couleur de base
\begin{EnvQuadrillage}[NbCarreaux=30x5]
\end{EnvQuadrillage}

%36x8 petits carreaux, avec débordements 3/3, couleur LightSteelBlue
\begin{EnvQuadrillage}[NbCarreaux=36x8,Elargir=3/3]<LightSteelBlue>
\end{EnvQuadrillage}

%15x5 grand carreaux, avec débordements 1/2, couleur orange, centré
\begin{center}
	\begin{EnvQuadrillage}[NbCarreaux=15x5,Elargir=1/2,Seyes]<orange>
	\end{EnvQuadrillage}
\end{center}
\end{PresentationCode}

\medskip

\begin{EnvQuadrillage}[NbCarreaux=30x5,Cadre]
\end{EnvQuadrillage}

\smallskip

\begin{EnvQuadrillage}[NbCarreaux=36x8,Elargir=3/3,Cadre]<LightSteelBlue>
\end{EnvQuadrillage}

\begin{center}
	\begin{EnvQuadrillage}[NbCarreaux=15x5,Elargir=1/2,Seyes,Cadre]<orange>
\end{EnvQuadrillage}
\end{center}

\pagebreak

\subsection{Écrire sur les lignes}

L'idée est maintenant de pouvoir écrire sur les lignes du quadrillage créé, et pour garantir le fait d'écrire \textit{pile} sur le ligne, on applique les recommandations suivantes :

\begin{itemize}
	\item les lignes doivent être saisies une par une, du \og haut \fg{} vers le \og bas \fg{} ;
	\item pas de paragraphe \textit{multilignes}, pas de puce ou de numérotation ;
	\item une ligne peut être passée.
\end{itemize}

\begin{PresentationCode}{listing only}
...
	%pour saisir les lignes, une par une
	\EcrireLigne[clés]<alignement>{texte}
	%pour passer la ligne
	\PasseLigne
...
\end{PresentationCode}

Le premier argument, entre \texttt{[...]} propose les \Cle{clés} :

\begin{itemize}
	\item \Cle{DecalH} pour spécifier le décalage horizontal (en carreaux) de la ligne, par rapport à la \textcolor{orange}{marge} ; \hfill~défaut : \Cle{0}
	\item \Cle{DecalV} pour spécifier le décalage vertical du texte par rapport à la ligne ; \hfill~défaut : \Cle{0pt}
	\item \Cle{Echelle} pour spécifier l'échelle du texte à écrire.\hfill~défaut : \Cle{1}
\end{itemize}

Le deuxième argument, \textit{optionnel}, entre \texttt{<...>} permet de spécifier l'alignement horizontal (parmi \Cle{left/center/right}) du texte dans le \textcolor{ForestGreen}{\textit{cadre}} de base, \Cle{left} par défaut.

\medskip

Le troisième argument, \textit{mandataire} et entre \texttt{\{...\}} est quant à lui le texte à saisir, avec possibilité de spécifier taille, couleur, fonte, etc

\begin{PresentationCode}{listing only}
\begin{EnvQuadrillage}[NbCarreaux=36x8]
	\EcrireLigne{mon texte sur la ligne 1\ldots}
	\EcrireLigne<center>{\ttfamily mon texte, en fonte teletype augmentée de 25\,\%, centré sur la ligne 2\ldots}
	\EcrireLigne<right>{mon texte aligné à droite sur la ligne 3\ldots}
	\EcrireLigne[DecalV=0.1]{\textcolor{red}{mon texte rouge sur la ligne 4, décalé de 1mm vers le haut\ldots}}
	\PasseLigne
	\EcrireLigne[Echelle=0.5]{\sffamily mon texte, en fonte sans réduite de 50\,\%, sur la ligne 6\ldots}
	\EcrireLigne[DecalH=3]{\cursive mon texte sur la ligne 7, décalé de 3 carreaux\ldots}
\end{EnvQuadrillage}
\end{PresentationCode}

\begin{EnvQuadrillage}[NbCarreaux=36x8]
	\EcrireLigne{mon texte sur la ligne 1\ldots}
	\EcrireLigne<center>{\ttfamily mon texte, en fonte teletype augmentée de 25\,\%, centré sur la ligne 2\ldots}
	\EcrireLigne<right>{mon texte aligné à droite sur la ligne 3\ldots}
	\EcrireLigne[DecalV=0.1]{\textcolor{red}{mon texte rouge sur la ligne 4, décalé de 1mm vers le haut\ldots}}
	\PasseLigne
	\EcrireLigne[Echelle=0.5]{\sffamily mon texte, en fonte sans réduite de 50\,\%, sur la ligne 6\ldots}
	\EcrireLigne[DecalH=3]{\cursive mon texte sur la ligne 7, décalé de 3 carreaux\ldots}
\end{EnvQuadrillage}

\begin{PresentationCode}{listing only}
\begin{EnvQuadrillage}[NbCarreaux=22x4,Marge=1,Elargir=2/3,Seyes,Cadre]
	\EcrireLigne[Echelle=1.5]{\textcolor{red}{mon texte rouge, un peu agrandi,sur la ligne 1\ldots}}
	\EcrireLigne[Echelle=1.15,DecalH=1]{$(1+x)^2=1+2x+x^2$ sur la l2, avec un décal de 1 en plus de la marge\ldots}
	\EcrireLigne[DecalH=-1]{\textcolor{blue}{mon texte bleu, remis à gauche, sur la ligne 3\ldots}}
\end{EnvQuadrillage}

\end{PresentationCode}
\begin{EnvQuadrillage}[NbCarreaux=22x4,Marge=1,Elargir=2/3,Seyes,Cadre]
	\EcrireLigne[Echelle=1.5]{\textcolor{red}{mon texte rouge, un peu agrandi,sur la ligne 1\ldots}}
	\EcrireLigne[Echelle=1.15,DecalH=1]{$(1+x)^2=1+2x+x^2$ sur la ligne 2, avec un décalage de 1 carreau en plus de la marge\ldots}
	\EcrireLigne[DecalH=-1]{\textcolor{blue}{mon texte bleu, remis à gauche, sur la ligne 3\ldots}}
\end{EnvQuadrillage}

\newpage

\part*{Historique}

\verb|v0.1.0|~:~~~~Version initiale

\end{document}